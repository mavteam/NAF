
\section{Opinion layer}
\label{sec:opinion-layer}

The sentiment related information in NAF is represented at two levels: the
(already existing) term layer and a (new) opinion layer. The term layer
represents information copied from an external source like the sentiment
lexicon. Information can be stored at both word and sense-synset
level. Sentiment information in the term layer is the building block for the
sentiment analysis tool which then tries to generate information needed to
fill the slots (opinion holder, opinion target, opinion epression) of an
opinion triplet. These triplets are represented in the opinion
layer. Starting from these triplets, final sentiment analysis results can be
calculated.

Basic forms of sentiment analysis where polarity is aggregated at sentence
or document level may ignore the opinion layer triplets and make use of the
term level sentiment only. In these cases (for example, polarity
classification of product reviews) , opinion target (i. e. the product) and
opinion holder (i. e. the reviewer) are known on beforehand. However, when
more complex sentiment analysis is performed, the opinion triplets are
needed. An example of such an analysis, is product feature-based SA where
the target of the opinion is not just the product but some specific feature
of the product. Another example concerns the analysis of , for example, news
or blogs where different people (i. e. opinion holders) may have different
opinions about different targets. In those cases, aggregation of the
polarity values found in the text, is not enough.\\


The \texttt{<opinion>} element has one attribute:
\begin{itemize}
\item \texttt{id} (\textbf{required}): the unique identifier of the opinion,
  starting with the prefix ``o''.
\end{itemize}

The \texttt{<opinion>} element contains of the following subelements:
\begin{itemize}
\item \texttt{<opinion\_holder>}: the holder of the opinion, that is, the
  speaker or some actor in the text.
\item \texttt{<opinion\_target>} :  the target of the opinion (about what).
\item \texttt{<opinion\_expression>}: the opinion expression, which spans the
  terms comprising the opinion proper.
\end{itemize}
Each of these elements contain a \texttt{<span>} sub-element spanning terms
(see Section \ref{sec:span}).\\

The \texttt{<opinion\_holder>} element has the following attributes:
\begin{itemize}
\item \texttt{type} (optional): the type of the holder (for instance,
  ``Speaker/Writer'').
\end{itemize}

The \texttt{<opinion\_expression>} element has the following attributes:
\begin{itemize}
\item \texttt{polarity} (optional): refers to the positive or negative
  orientation of the expression.
\item \texttt{strength}: refers to the strength of the expression.
\item \texttt{subjectivity}: refers to whether an expression is subjective
  or not.
\item \texttt{sentiment\_semantic\_type}: refers to sentiment related
  semantic types like emotion, judgment, belief, speculation.
\item \texttt{sentiment\_product\_feature} : refers to specific features of
  entities, to be used in feature/aspect-based sentiment analysis.
\end{itemize}

% The fillers of the three elements (holder, target, type) can be found in one
% sentence or in different ones (cf. ex. 1). Moreover, one sentence can
% contain multiple opinion triplets (cf. ex. 2).


% Example

% (1)
% Vicky Masing,"http://twitter.com/VickyMasing/statuses/245094342799286273",2012-09-10T10:13:24Z,"english","Fountain of Bellagio Las Vegas. One has got to witness that magical show. Definitely beautiful, definitely mesmerizing."

% opinion expression: Definitely beautiful, definitely mesmerizing.
% polarity=positive, strength=strong, sentiment_semantic_type=evaluation,
% sentiment_product_feature=""
% opinion holder : Speaker/Writer
% opinion target : Fountain of Bellagio Las Vegas

% (2)
% We stayed at The Toren as a transit hotel for our trip from saudi arabia to USA, the hotel was just amazing in all aspects,
% the service was great, the people working their were really helpful and nice, the food at the breakfast was delicious and
% fresh and the location is great, very near to tram stops, the central station, Shopping...

% opinion expression:             really helpful and nice
% polarity:                               positive 
% strength:                               strong 
% sentiment_semantic_type:                evaluation
% sentiment_product_feature:      staff
% opinion holder :                        Speaker/Writer
% opinion target :                        people working there

% opinion expression:             delicious
% polarity:                               positive
%  strength:                      strong
%  sentiment_semantic_type:               evaluation
% sentiment_product_feature:      breakfast
% opinion holder :                        Speaker/Writer
% opinion target :                        the food at the breakfast

% (N.B. there are more opinion triplets in this fragment than the ones given here).


NAF Example:

\begin{Verbatim}[fontsize=\small]
<opinions>
  <!-- They had a nightmare with Hilton Hotel Paris. -->
  <opinion id="o1">
    <opinion_holder type="Speaker/Writer" >
      <span>
        <target id="t1"/>
      </span>
    </opinion_holder>
    <opinion_target>
      <span>
        <target id="t6"/>
        <target id="t7"/>
        <target id="t8"/>
      </span>
    </opinion_target>
    <opinion_expression polarity="negative" strength="strong" 
                        subjectivity="subjectivity"
                        sentiment_semantic_type="evaluation"
                        sentiment_product_feature="">
      <span>
        <target id="t3"/>
        <target id="t4"/>
      </span>
    </opinion_expression>
  </opinion>
</opinions>
\end{Verbatim}


%%% Local Variables: 
%%% mode: latex
%%% TeX-master: "naf"
%%% End: 
