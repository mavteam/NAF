
\section{Coreference}
\label{sec:coreference}

The coreference layer creates clusters of term spans (which we call
mentions) which share the same referent. For instance, ``London'' and ``the
capital city of England'' are two mentions referring to the same entity. It
is said that those mentions corefer.  A \texttt{<coref>} element represents
a mention cluster, and within \texttt{<coref>} each mention is represented
by a \texttt{<span>} element. One \texttt{<span>} within a \texttt{<coref>}
may have an attribute \texttt{primary} with value ``yes'', indicating that
this span is the most representative mention of the coreference
cluster\footnote{Typical candidates for representative mentions are the
  first Named Entity mention, first nominal mention, etc. }. Within each
\texttt{<span>} element, one \texttt{<target>} sub-element may have an
attribute \texttt{prinary} with value ``yes'' to represent the fact that
this particular term is the head of the mention. For instance, the head of
the mention ``the capital city of England'' is ``city''.

The \texttt{<coref>} element has the following attribute:
\begin{itemize}
\item \texttt{id} (\textbf{required}): unique id, starting with the prefix ``co''.
\item \texttt{type} (optional): type of the coreference set. It describes
  whether the coreference cluster refers to an entity instance, and event,
  etc.
\end{itemize}

As said before, \texttt{<coref>} element contains as many \texttt{<span>}
elements as elements in the coreference cluster (see section
\ref{sec:span} for details).

The \texttt{<coref>} element may also contain one
\texttt{<externalReferences>} element (see section
\ref{sec:external-references}), which links the coreference cluster to some
external entity (such as the lowest common subsumer synset in WordNet).

Example of a coreference cluster:

\begin{Verbatim}[fontsize=\small]
<coreferences>
  <coref id="co1" type="entity">
    <!-- London -->
    <span primary="yes"> <!-- London is the representative mention -->
      <target id="t12" head="yes"/>
    </span>
    <!-- the capital city of England -->
    <span>
      <target id="t1"/>
      <target id="t2"/>
      <target id="t3" head ="yes"/> <!-- city is the head -->
      <target id="t4"/>
      <target id="t5"/>
    </span>
  </coref>
</coreferences>
\end{Verbatim}


%%% Local Variables: 
%%% mode: latex
%%% TeX-master: "naf"
%%% End: 
