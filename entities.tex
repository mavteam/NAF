
\section{Named Entities}
\label{sec:named-entities}

Named entities are information units such as names, including person,
organization and location names, and numeric expressions including time,
date, money and percent expressions. NAF describes named entity mentions
present in text in a separate layer (\texttt{<entities>}), described in this
section. The \texttt{<entity>} element is used to represent a named entity
in the document. It may be used either for referencing single mentioned
entities or multi-mentioned entities: in the latter case, the element
contains clusters of term spans, which we call mentions, each mention
referencing the same entity. The text anchor is described by means of the
\texttt{<span>} elements, which always refer to term
elements. \texttt{<entity>} elements also annotates the type of the entity
mention. Besides, it can be linked to an external resource (such as
Wikipedia) using the \texttt{<externalRef>} element.

The \texttt{<entity>} element has the following attributes:
\begin{itemize}
\item \texttt{id} (\textbf{required}): unique id, starting with the prefix ``e''.
\item \texttt{type} (optional): named entity type. Some possible values:
  \begin{itemize}
  \item \texttt{Time}
  \item \texttt{Location}
  \item \texttt{Organization}
  \item \texttt{Person}
  \item \texttt{Money}
  \item \texttt{Percent}
  \item \texttt{Date}
  \item \texttt{Misc}
\end{itemize}
\item \texttt{source} (optional): the name of the process which created the
  entity.
\end{itemize}

The \texttt{<entity>} element has the following sub-elements:
\begin{itemize}
\item \texttt{<references>} (\textbf{required}): this element contains one
  or more \texttt{<span>} element, spanning terms.
\item \texttt{externalReferences} (optional): contains one or more
  \texttt{<externalRef>} elements.
\end{itemize}

A \texttt{<span>} sub-element can be used to reference the different
occurrences of the same named entity in the document or mentions of it,
using \texttt{<target>} elements to refer to the terms it spans. If the
entity is composed by multiple words, multiple target elements are used. See
Section \ref{sec:span}.

The optional \texttt{<externalRef>} sub-element is used to associate terms
to external lexical or semantic resources. See Section
\ref{sec:external-references}.

Example of an entity mention:

\begin{Verbatim}[fontsize=\small]
<entities>
  <!-- John -->
  <entity id="e1" type="person">
    <references>
      <span>
        <target id="t1"/>
      </span>
    </references>
  </entity>
  <!-- New York -->
  <entity id="e2" type="location" source="spotlight-v07">
    <references>
      <span>
        <target id="t8"/>
      </span>
    </references>
    <externalReferences>
      <externalRef resource="Wikipedia"
                   reference="New_York"
                   confidence="0.85"/>
      <externalRef resource="Wikipedia"
                   reference="New_York,Lincolnshire"
                   confidence="0.15"/>
    </externalReferences>
  </entity>
  <!-- London -->
  <entity id="e3" type="location">
    <references>
      <!-- London -->
      <span>
        <target id="t12" head="yes"/>
      </span>
      <!-- The capital city of England -->
      <span>
        <target id="t1" />
        <target id="t2" />
        <target id="t3" head="yes"/> <!-- city is the head -->
        <target id="t4" />
        <target id="t5" />
      </span>
    </references>
    <externalReferences>
      <externalRef reference="ref01011020" 
                   resource="GeoNames"/>
      <externalRef reference="http://wwww.wikipedia.com/London" 
                   resource="Wikipedia"/>
    </externalReferences>
  </entity>
</entities>
\end{Verbatim}


%%% Local Variables: 
%%% mode: latex
%%% TeX-master: "naf"
%%% End: 
